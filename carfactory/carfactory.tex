\documentclass[UTF8,18pt]{ctexart}
\usepackage{geometry}
\usepackage{url}

% to fix url issue
\def\UrlBreaks{%
	\do\/%
	\do\a\do\b\do\c\do\d\do\e\do\f\do\g\do\h\do\i\do\j\do\k\do\l%
	\do\m\do\n\do\o\do\p\do\q\do\r\do\s\do\t\do\u\do\v\do\w\do\x\do\y\do\z%
	\do\A\do\B\do\C\do\D\do\E\do\F\do\G\do\H\do\I\do\J\do\K\do\L%
	\do\M\do\N\do\O\do\P\do\Q\do\R\do\S\do\T\do\U\do\V\do\W\do\X\do\Y\do\Z%
	\do0\do1\do2\do3\do4\do5\do6\do7\do8\do9\do=\do/\do.\do:%
	\do\*\do\-\do\~\do\'\do\"\do\-}
\Urlmuskip=0mu plus 0.1mu

\usepackage{diagbox}
\geometry{a4paper,centering,scale=0.8}
\usepackage[format=hang,font=small,textfont=it]{caption}
\usepackage{graphicx}
\usepackage{authblk}
\usepackage{float}
\usepackage{amsmath}
\usepackage{indentfirst}
\usepackage{minted}
\setlength{\parindent}{2em}
\newenvironment{myquote}
{\begin{quote}\kaishu\zihao{-5}}
	{\end{quote}}
\title{\heiti C++ 实验:继承和多态}
\author{\kaishu 郭XX}
\affil{\small{xxxxxxxxx,物联网 xxxx,XX 大学}}
\date{\today}
\bibliographystyle{plain}

\begin{document}
\maketitle
% 摘要部分
\begin{abstract}
这是使用\LaTeX 写的实验C++报告:类的继承以及多态。
\end{abstract}
{\small 关键词:\underline{C++},\underline{类继承},\underline{多态},\underline{虚函数}}

% 目录部分
\tableofcontents
\section{C++ 继承}
\subsection{简介}
面向对象程序设计中最重要的一个概念是继承。继承允许我们依据另一个类来定义一个类,这使得创建和维护一个应用程序变得更容易。这样做,也达到了重用代码功能和提高执行时间的效果。

当创建一个类时,您不需要重新编写新的数据成员和成员函数,只需指定新建的类继承了一个已有的类的成员即可。这个已有的类称为\textbf{基类},新建的类称为\textbf{派生类}。

继承代表了 \textbf{is a }关系。例如,哺乳动物是动物,狗是哺乳动物,因此,狗是动物,等等。
\subsection{继承类型}
这里也是引用在线【 菜鸟教程】 的说明:
\begin{myquote}
当一个类派生自基类,该基类可以被继承为 \textbf{public}、\textbf{protected} 或 \textbf{private} 几种类型。继承类型是通过上面讲解的访问修饰符 access-specifier 来指定的。

我们几乎不使用 \textbf{protected }或 \textbf{private }继承,通常使用 \textbf{public} 继承。当使用不同类型的继承时,遵循以下几个规则:

\textbf{公有继承(public)}:当一个类派生自\textbf{公有}基类时,基类的\textbf{公有}成员也是派生类的\textbf{公有}成员,基类的\textbf{保护}成员也是派生类的\textbf{保护}成员,基类的\textbf{私有}成员不能直接被派生类访问,但是可以通过调用基类的\textbf{公有}和\textbf{保护}成员来访问。

\textbf{保护继承(protected)}: 当一个类派生自\textbf{保护}基类时,基类的\textbf{公有}和\textbf{保护}成员将成为派生类的\textbf{保护}成员。

\textbf{私有继承(private)}:当一个类派生自\textbf{私有}基类时,基类的\textbf{公有}和\textbf{保护}成员将成为派生类的\textbf{私有}成员。
\end{myquote}
\subsection{访问控制和继承}
派生类可以访问基类中所有的非私有成员。因此基类成员如果不想被派生类的成员函数访问,则应在基类中声明为 private。

我们可以根据访问权限总结出不同的访问类型,如 表\ref{tb:xianzhi} 所示:
\begin{table}[H]
	\centering
	\begin{tabular}{|c|*{3}{c}|}
		\hline
		访问&public&protected&private \\
		\hline
		同一个类&Yes&Yes&Yes \\
		派生类&Yes&Yes&No \\
		外部的类&Yes&No&No \\
		\hline
	\end{tabular}
	\caption{上图演示了不同的访问类型。}
	\label{tb:xianzhi}
\end{table}

\textbf{派生类无法继承的情况:}

\qquad1. 基类的构造函数、析构函数和拷贝构造函数。

\qquad2. 基类的重载运算符。

\qquad3. 基类的友元函数。

\subsection{基类和派生类构建过程}
在 C++ Primer Plus 6th 中讲解如下:
\begin{myquote}
派生类构造函数:

\qquad1. 首先创建基类对象

\qquad2. 派生类构造函数应通过成员初始化列表传递给基类构造函数

\qquad3. 派生类构造函数应初始化派生类新增数据成员
\end{myquote}
\section{C++ 多态}

\subsection{什么是多态}
多态按字面的意思就是多种形态。当类之间存在层次结构,并且类之间是通过继承关联时,就会用到多态。也可解释为派生类和基类的行为是不同的,使用同一个方法行为要根据上下文确定。

\subsection{如何实现多态公有继承}
有两种方法可以实现多态公有继承:

\qquad1. 在派生类重新定义基类 

\qquad2. 使用虚方法 

一般我们使用第二种,虚方法,又叫虚函数。
\subsubsection{纯虚函数以及派生类}

\textbf{纯虚函数:}
纯虚函数或纯虚方法是一个虚拟函数,需要由不是抽象的派生类来实现。

\textbf{派生类:}
在编程语言中,抽象类型是一种不能直接实例化的主导类型系统中的类型;一个不抽象的类型可以被实例化被称为具体类型。
\subsubsection{如何通过纯虚函数设计派生类}
设计抽象类(通常称为 ABC)的目的,是为了给其他类提供一个可以继承的适当的基类。抽象类不能被用于实例化对象,它只能作为接口使用。如果试图实例化一个抽象类的对象,会导致编译错误。
因此,如果一个 ABC 的子类需要被实例化,则必须实现每个虚函数,这也意味着 C++ 支持使用 ABC 声明接口。如果没有在派生类中重载纯虚函数,就尝试实例化该类的对象,会导致编译错误。
可用于实例化对象的类被称为具体类。
如果类中至少有一个函数被声明为纯虚函数,则这个类就是抽象类。纯虚函数是通过在声明中使用 "= 0" 来指定的。
\section{实验目的}
\subsection{简介}
通过学习简单的类定义和类继承,了解C++的多态,抽象类,虚函数;派生类。实现数据和程序的方便可视的处理和计算。

此外我是使用了C++11标准的成员初始化器列表 \mintinline{c}{member initializer} ,以及基于范围的 \mintinline{c}{for} 循环 for range用来熟悉新版C++特性,学习更多,跟随潮流。

具体请看代码赏析。

这是我们最后一个C++实验,也是我用\LaTeX 写的第二篇实验报告,之所以用 \LaTeX 写实验报告,是因为我想认真完成,也想试试新的东西。我觉得自己喜欢的事业,就应该应该的认真完成,虽然十分花费时间,但是我觉得很值得,年轻就应该多闯闯。

最后谢谢老师的帮助和教导,老师您幸苦了。

\section{附录}
\subsection{源码}

\subsubsection{main.cpp}
\begin{minted}[%showtabs,
			   obeytabs,
               linenos,
               numbersep=5pt,
               breaksymbolleft={\ },
               breaklines, %自动折行  
               tabsize=4, %设置tab空格数 
               frame=lines,
               framesep=2mm]{cpp}
//Sources:main.cpp
#include <iostream>
#include "sportCar.h"
#include "Truck.h"
#include "Car.h"

using namespace std;

int main()
{
    //Car mycar();
    //The class car is abstract,so it can not be instantiated.

    //sportCar e1();
    //e1.print();
    //this is error,i do not know why.

    sportCar car1("White");
    car1.print();
    std::cout<<std::endl;
    sportCar car2("Red",4,4);
    car2.print();
    std::cout<<std::endl;
    //Truck e2();
    //e2.print();
    //this also is error,i do not know why.

    Truck car3("Gray");
    car3.print();
    std::cout<<std::endl;

    Truck car4("Black",6,"Gasoline");
    car4.print();
    std::cout<<std::endl;

    Car *ptr[4];
    ptr[0]=&car1;
    ptr[1]=&car2;
    ptr[2]=&car3;
    ptr[3]=&car4;

    //C++11 for range
    for( const auto &y : ptr )
    {   // Type inference by const reference.
        // Observes in-place. Preferred when no modify is needed.
        // 参考:https://docs.microsoft.com/zh-cn/cpp/cpp/range-based-for-statement-cpp
        y->print();
    }
    /*
        以上代码等价:
        for (int i = 0; i < 5; i++)
            ptr[i]->print();
    */

    cout << endl;

    return 0;
}
#endif
\end{minted}
\subsubsection{car.cpp}
\begin{minted}[%showtabs,
			   obeytabs,
               linenos,
               numbersep=5pt,
               breaksymbolleft={\ },
               breaklines, %自动折行  
               tabsize=4, %设置tab空格数 
               frame=lines,
               framesep=2mm]{cpp}
//Sources:Car.cpp
#include "Car.h"

void Car::setColor(std::string color)
{
    Color=color;
}

void Car::setTire(int tire)
{
    Tire=tire;
}

std::string Car::getColor() const
{
    return Color;
}

int Car::getTire() const
{
    return Tire;
}

void Car::print()
{
    std::cout<<"The car's color is "<<Color<<" ,It has "<<Tire<<" tire(s)."<<std::endl;
}

\end{minted}
\subsubsection{car.h}
\begin{minted}[%showtabs,
			   obeytabs,
               linenos,
               numbersep=5pt,
               breaksymbolleft={\ },
               breaklines, %自动折行  
               tabsize=4, %设置tab空格数 
               frame=lines,
               framesep=2mm]{cpp}
//Headers:Car.h
#ifndef CAR_H
#define CAR_H

#include <iostream>
#include <string>

/*
    设计抽象类(通常称为 ABC)的目的,是为了给其他类提供一个可以继承的适当的基类。抽象类不能被用于实例化对象,它只能作为接口使用。如果试图实例化一个抽象类的对象,会导致编译错误。
    因此,如果一个 ABC 的子类需要被实例化,则必须实现每个虚函数,这也意味着 C++ 支持使用 ABC 声明接口。如果没有在派生类中重载纯虚函数,就尝试实例化该类的对象,会导致编译错误。
    可用于实例化对象的类被称为具体类。
    如果类中至少有一个函数被声明为纯虚函数,则这个类就是抽象类。纯虚函数是通过在声明中使用 "= 0" 来指定的
    摘自:http://www.runoob.com/cplusplus/cpp-interfaces.html
*/


/*
    从一个类派生出一个类,原始类成为基类,继承类称为派生类
    多态:派生类和基类的行为是不同的,同一个方法行为根据上下文,有两种方法可以实现多态公有继承:
    1.在派生类重新定义基类
    2.使用虚方法
    - C++ Primer Plus 6th
*/

class Car
{
    //这是个抽象类
public:
	Car(std::string color="White",int tire=4) : Color(color),Tire(tire){}; //constuctor 构造函数
	/*
       member initializer 成员初始化器列表
       等价于:
       Car(std::string color="White",int tire=4)
       {
            Color=color;
            Tire=tire;
       }
       - https://www.geeksforgeeks.org/when-do-we-use-initializer-list-in-c/
	*/
	~Car(){};//destructors
	void setColor(std::string);
	std::string getColor() const;
	void setTire(int);
	int getTire() const;
	virtual void print()=0;//纯虚函数
    /*
        A pure virtual function or pure virtual method is a virtual function that is required to be implemented by a derived class that is not abstract" - Wikipedia
        “纯虚函数或纯虚方法是一个虚拟函数,需要由不是抽象的派生类来实现“ - 维基百科
        In programming languages, an abstract type is a type in a nominative type system that cannot be instantiated directly; a type that is not abstract – which can be instantiated – is called a concrete type. Every instance of an abstract type is an instance of some concrete subtype. Abstract types are also known as existential types.
        “在编程语言中,抽象类型是一种不能直接实例化的主导类型系统中的类型;一个不抽象的类型可以被实例化被称为具体类型。”- 维基百科
        C++ Virtual/Pure Virtual Explained:
        https://stackoverflow.com/questions/1306778/c-virtual-pure-virtual-explained
        Virtual function:
        https://en.wikipedia.org/wiki/Virtual_function
    */
private:
	std::string Color;
	int Tire;
};//do not forget ;

#endif
\end{minted}
\subsubsection{sportcar.cpp}
\begin{minted}[%showtabs,
			   obeytabs,
               linenos,
               numbersep=5pt,
               breaksymbolleft={\ },
               breaklines, %自动折行  
               tabsize=4, %设置tab空格数 
               frame=lines,
               framesep=2mm]{cpp}
//Sources:sportCar.cpp
#include "sportCar.h"

void sportCar::setDoor(int door)
{
    Door=door;
}

int sportCar::getDoor() const
{
    return Door;
}

void sportCar::print()
{
    //std::cout<<"The sportCar's color is "<<Color<<" ,It has "<<Tire<<" tire(s) and "<<Door<<" Door(s)."<<std::endl;
    //This is error,the complier will tell it is private,but we can use like this
    int tire=Car::getTire();
    std::string color=Car::getColor();
    std::cout<<"The sportCar's color is "<<color<<" ,It has "<<tire<<" tire(s) and "<<Door<<" door(s)."<<std::endl;
    //or like this:
    Car::print();
    std::cout<<"It also has "<<Door<<" door(s)."<<std::endl;
}

\end{minted}
\subsubsection{sportcar.h}
\begin{minted}[%showtabs,
			   obeytabs,
               linenos,
               numbersep=5pt,
               breaksymbolleft={\ },
               breaklines, %自动折行  
               tabsize=4, %设置tab空格数 
               frame=lines,
               framesep=2mm]{cpp}
//Headers:sportCar.h
#ifndef SPORTCAR_H
#define SPORTCAR_H

#include "Car.h"

class sportCar : public Car
{
//继承Car类
public:
	sportCar(std::string color="Black", int tire=4, int door=2): Car(color,tire),Door(door) {};
	//将参数从派生类构造函数传递给基类构造函数
    /*
        派生类构造函数
        首先创建基类对象
        派生类构造函数应通过成员初始化列表传递给基类构造函数
        派生类构造函数应初始化派生类新增数据成员
        - C++ Primer Plus 6th
    */
	~sportCar(){};
	void setDoor(int);
	int getDoor() const;
	void virtual print();
private:
	int Door;
};


#endif

\end{minted}

\subsubsection{Truck.cpp}
\begin{minted}[%showtabs,
			   obeytabs,
               linenos,
               numbersep=5pt,
               breaksymbolleft={\ },
               breaklines, %自动折行  
               tabsize=4, %设置tab空格数 
               frame=lines,
               framesep=2mm]{cpp}
//Sources:Truck.cpp
#include "Truck.h"

Truck::Truck(std::string color,int tire,std::string engine)
    : Car(color,tire)
{
    setEngine(engine);
    // Engine=engine;
}
Truck::~Truck()
{

}
void Truck::setEngine(std::string engine)
{
    Engine=engine;
}
std::string Truck::getEngine()
{
    return Engine;
}
void Truck::print()
{
    int tire=Car::getTire();
    std::string color=Car::getColor();
    std::cout<<"The truck's color is "<<color<<" ,It has "<<tire<<" tire(s) and "<<Engine<<" engine."<<std::endl;
    //or like this:
    Car::print();
    std::cout<<"It also has "<<Engine<<" engine."<<std::endl;
}               
\end{minted}             
\subsubsection{Truck.h}
\begin{minted}[%showtabs,
			   obeytabs,
               linenos,
               numbersep=5pt,
               breaksymbolleft={\ },
               breaklines, %自动折行  
               tabsize=4, %设置tab空格数 
               frame=lines,
               framesep=2mm]{cpp}
//Headers:Truck.h
#ifndef TRUCK_H
#define TRUCK_H

#include "Car.h"

class Truck : public Car
{
public:
	Truck(std::string="",int=4,std::string="Diesel");
	~Truck();
	void setEngine(std::string);
	std::string getEngine();
	void virtual print();
private:
	std::string Engine;
};



#endif
\end{minted}

\subsection{UML图}
\begin{figure}[H]
	\centering
	\includegraphics[scale=0.8]{uml.pdf}
	\caption{ UML 图}
	\label{fig:uml}
\end{figure}
\bibliography{computer}
\nocite{LiuHaiYang}
\nocite{cpp}
\nocite{for_range}
\nocite{xuhanshu}
\nocite{jicheng}
\nocite{jiekou}
\end{document}